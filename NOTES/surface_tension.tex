\documentclass[10pt,a4paper]{article}

\bibliographystyle{plain}
\usepackage{latexsym}
\usepackage{algorithm,algorithmic}
\usepackage{amssymb,amsthm}
\usepackage{amsmath,amsfonts}
\usepackage{mathrsfs}
\usepackage{enumerate}
\usepackage{graphicx,psfrag}
\usepackage{comment}
%\usepackage[notcite,notref]{showkeys}

\newtheorem{theorem}{Theorem}[section]
\newtheorem{remark}{Remark}[section]
\newtheorem{lemma}{Lemma}[section]
\newtheorem{corollary}{Corollary}[section]
\newtheorem{assumption}{Assumption}[section]
\newtheorem{definition}{{Definition}}[section]
\newtheorem{proposition}{Proposition}[section]
\newtheorem{question}{Question}


\author{G\"unay Do\u gan}
\title{Computing Surface Tension}

\begin{document}

\maketitle

%%%%%%%%%%%%%%%%%%%%%%%%%%%%%%%%%%%%%%%%%%%%%%%%%%%%%%%%%%%
\section{Introduction}
%%%%%%%%%%%%%%%%%%%%%%%%%%%%%%%%%%%%%%%%%%%%%%%%%%%%%%%%%%%

Our goal is to compute a displacement field $\vec{u}$ that minimizes
the energy
%
\begin{equation}\label{E:energy}
E(\vec{u}) = \int_\mathcal{D} \left( \frac{1}{2}\sigma_{ij}\varepsilon_{ij} 
-\vec{f}\cdot\vec{u} \right) dx + \gamma \int_{\Gamma(\vec{u})} d\Gamma.
\end{equation}
%
The energy \eqref{E:energy} has two components: the elastic energy and 
the surface tension, which is given by the surface area of 
the displaced surface
%
\begin{equation}\label{E:displaced-surface}
\Gamma(\vec{u}) = \{\vec{x} + \vec{u}(\vec{x}), \vec{x} \in \Gamma_0 \}.
\end{equation}
%
To find a minimizer of the energy \eqref{E:energy}, we take 
its first variation $\delta E(\vec{u};\vec{\phi})$ 
with respect to a test function $\vec{\phi}$ and seek 
$\vec{u}$ that satisfies
\[
\delta E(\vec{u};\vec{\phi}) = 0, \quad \forall \vec{\phi}
\]
or more explicitly
%
\begin{equation}\label{E:1st-variation}
\int_\mathcal{D} \left( \frac{1}{2} D\vec{u}:C D\vec{\phi} 
- \vec{f}\cdot\vec{\phi} \right) dx 
+ \gamma \int_{\Gamma(\vec{u})} \kappa \vec{\phi}\cdot\hat{n} \, d\Gamma = 0,
\quad \forall \vec{\phi}.
\end{equation}
%
Given the first variation \eqref{E:1st-variation}, we could
attempt to write an iterative scheme to compute $\vec{u}$
that satisfies, for example,
%
\begin{equation}\label{E:1st-iterative-scheme}
\int_\mathcal{D} \left( \frac{1}{2} D\vec{u}^{n+1}:C D\vec{\phi} 
- \vec{f}\cdot\vec{\phi} \right) dx 
= -\gamma \int_{\Gamma(\vec{u}^n)} \kappa \vec{\phi}\cdot\hat{n} \, d\Gamma,
\quad \forall \vec{\phi}.
\end{equation}
%
In this way, we separate the linear and nonlinear parts
and treat the nonlinear part explicitly without the need
for a nonlinear solve at each step of iterations.
To use this scheme, we would still need to verify 
that the scheme converges to a solution. 
There are other issues that need to be addressed 
as well. The right hand side in \eqref{E:1st-iterative-scheme}
is an integral computed on the displaced surface
$\Gamma(\vec{u})$. Therefore, at each iteration, 
we need to move the surface $\Gamma$ to its displaced
position and then evaluate the surface integral. 
Moreover, evaluating the integral requires computing
the curvature of the displaced curve $\Gamma(\vec{u})$.
Ideally we would like to avoid both tracking the displaced
curve and computing its curvature. For this reason,
in the following, we rederive the first variation
of the surface tension term in \eqref{E:energy}. 
Our goal is to obtain an expression that is
evaluated on the initial surface $\Gamma_0$
and does not require the curvature of the surface.
For the moment, we restrict our attention 
to parametric planar curves in 2d. 


%%%%%%%%%%%%%%%%%%%%%%%%%%%%%%%%%%%%%%%%%%%%%%%%%%%%%%%%%%%
\section{First variation of curve length}
%%%%%%%%%%%%%%%%%%%%%%%%%%%%%%%%%%%%%%%%%%%%%%%%%%%%%%%%%%%

We start by taking the first variation of
%
\begin{equation}\label{E:curve-length}
E(\Gamma_0) = \int_{\Gamma_0} d\Gamma = \int_{s_1}^{s_2} |\Gamma_0'(s)|ds
\end{equation}
%
with respect to a given parametric curve
\[
\Gamma_0(s) = \left( x(s), y(s) \right), \quad s_1 \leqslant s \leqslant s_2.
\]
Let us write \eqref{E:curve-length} more explicitly
%
\begin{align*}
\Gamma_0'(s)&=\left( x'(s), y'(s) \right), \\
|\Gamma_0'(s)|&=\left( |x'(s)|^2 + |y'(s)|^2 \right)^\frac{1}{2}
\end{align*}
%
\[
\Rightarrow  E(\Gamma_0) = \int_{s_1}^{s_2} \left( |x'(s)|^2 + |y'(s)|^2 \right)^\frac{1}{2} ds.
\]
We define the perturbed curve
\[
\Gamma_\varepsilon(s) = \left( x(s)+\varepsilon \phi(s), y(s) + \varepsilon \psi(s) \right)
\]
and compute
\[
\delta E(\Gamma_0) = \lim_{\varepsilon \rightarrow 0} \frac{E(\Gamma_\varepsilon) - E(\Gamma)}{\varepsilon}
= \frac{d}{d\varepsilon} E(\Gamma_\varepsilon) |_{\varepsilon=0}.
\]
We need to take the derivative of
\[
E(\Gamma_\varepsilon) = \int_{s_1}^{s_2} \left( |x'(s)+\varepsilon \phi'(s)|^2 + |y'(s) + \varepsilon \psi'(s)|^2 \right)^\frac{1}{2} ds
\]
with respect to $\varepsilon$. We start with
%
\begin{equation*}
\frac{d}{d\varepsilon} |\Gamma_\varepsilon'| 
= \frac{(x'(s)+\varepsilon \phi'(s)) \phi'(s) + (y'(s) + \varepsilon \psi'(s))\psi'(s) }{\left( |x'(s)+\varepsilon \phi'(s)|^2 + |y'(s) + \varepsilon \psi'(s)|^2 \right)^{\frac{1}{2}}},
\end{equation*}
%
and
%
\begin{equation*}
\frac{d}{d\varepsilon} |\Gamma_\varepsilon'|_{\varepsilon=0} = \frac{x'(s)\phi'(s) + y'(s)\psi'(s)}{\left( |x'(s)|^2 + |y'(s)|^2 \right)^{\frac{1}{2}}}.
\end{equation*}
%
We have
\[
\delta E(\Gamma_0) = \int_{s_1}^{s_2} \frac{x'(s)\phi'(s) + y'(s)\psi'(s)}{\left( |x'(s)|^2 + |y'(s)|^2 \right)^{\frac{1}{2}}} ds.
\]
Let us integrate $\phi', \psi'$ by parts (with the simplifying assumption that
$ \int_{s_1}^{s_2} = \oint$, that is, we integrate over a closed curve).
Start with
%
\begin{align*}
\left(  \frac{x'(s)}{\left( |x'(s)|^2 + |y'(s)|^2 \right)^{\frac{1}{2}}} \right)'
&=x''(s)\frac{1}{\left( |x'(s)|^2 + |y'(s)|^2 \right)^{\frac{1}{2}}} 
- x'(s)\frac{1}{2}\frac{(2x'(s)x''(s) + 2y'(s)y''(s))}{\left( |x'(s)|^2 + |y'(s)|^2 \right)^{\frac{3}{2}}} \\
&= \frac{x''\left( |x'|^2 + |y'|^2 \right) - x'x'x''-x'y'y''}{\left( |x'(s)|^2 + |y'(s)|^2 \right)^{\frac{3}{2}}} \\
&= -y'(s)\kappa(s) \\
&= -y' \frac{x'y'' - x''y'}{\left( |x'|^2 + |y'|^2 \right)^{\frac{3}{2}}} \\
\left(  \frac{y'(s)}{\left( |x'(s)|^2 + |y'(s)|^2 \right)^{\frac{1}{2}}} \right)' 
&= x' \frac{x'y'' - x''y'}{\left( |x'|^2 + |y'|^2 \right)^{\frac{3}{2}}} \\
&= x'(s)\kappa(s).
\end{align*}
%
\begin{align*}
\Rightarrow \delta E(\Gamma_0) &= \int \kappa(s) \left( -y'(s)\phi(s) + x'(s)\psi(s) \right) \frac{1}{|\Gamma_0'(s)|} |\Gamma_0'(s)| ds, \\
\Rightarrow \delta E(\Gamma_0) &= \int \kappa(s) \vec{\varphi}\cdot \hat{n}(s)|\Gamma_0'(s)| ds,
\end{align*}
where $\vec{\varphi} = (\phi(s),\psi(s))$ and $\hat{n} = \frac{1}{|\Gamma_0'(s)|}(-y'(s),x(s))$.

%%%%%%%%%%%%%%%%%%%%%%%%%%%%%%%%%%%%%%%%%%%%%%%%%%%%%%%%%%%
\subsection{Variation with respect to displacement}
%%%%%%%%%%%%%%%%%%%%%%%%%%%%%%%%%%%%%%%%%%%%%%%%%%%%%%%%%%%

Now we take the first variation of $E(\Gamma_u)$ with 
respect to the displacement field $\vec{u}$ (\emph{not $\Gamma_u$})
where $\Gamma_u$ is $\Gamma_0$ displaced by $\vec{u}$, that is,
\[
\Gamma_u(s) = \Gamma_0(s) + \vec{u}\cdot (x(s),y(s)) = \left( x(s)+u(x(s),y(s)), y(s)+v(x(s),y(s)) \right).
\]
We will need the derivative of $\Gamma_u(s)$
%
\begin{align*}
\Gamma_u'(s) &= \left( x' + \nabla u \cdot (x',y'), y' + \nabla v \cdot (x',y') \right) \\
\Gamma_u'(s) &= \Gamma_0'(s) + D\vec{u} \Gamma_0'(s) = (I + D\vec{u}) \Gamma_0'(s) \\
\Gamma_u'(s) &= \left( x' + \nabla u \cdot \Gamma_0', y' + \nabla v \cdot \Gamma_0' \right) .
\end{align*}
%
Note $D\vec{u} = \left( \begin{array}{c} \nabla u^T \\ \nabla v^T \end{array} \right)$. \\
Now we write
\[
|\Gamma_u'(s)| = \left( |x' + \nabla u \cdot \Gamma_0'|^2 + |y' + \nabla v \cdot \Gamma_0'|^2  \right)^\frac{1}{2}
\]
also the perturbed version (by $u \rightarrow u+\varepsilon\phi$, $v \rightarrow v+\varepsilon\psi$)
\[
|\Gamma_\varepsilon'(s)| = \left( |x' + \nabla u \cdot \Gamma_0'+ \varepsilon\nabla\phi \cdot \Gamma_0'|^2 + |y' + \nabla v \cdot \Gamma_0'+ \varepsilon\nabla\psi \cdot \Gamma_0'|^2  \right)^\frac{1}{2}.
\]
Differentiate $|\Gamma_\varepsilon'|$ with respect to $\varepsilon$ and evaluate at $\varepsilon=0$
%
\begin{equation*}
\frac{d}{d\varepsilon}|\Gamma_\varepsilon'| 
= \frac{1}{2}\left( \ldots \right)^\frac{1}{2}
2 \left( (x' + \nabla u \cdot \Gamma_0'+ \varepsilon\nabla\phi \cdot \Gamma_0')\nabla\phi\cdot\Gamma_0' +
(y' + \nabla v \cdot \Gamma_0'+ \varepsilon\nabla\psi \cdot \Gamma_0')\nabla\psi\cdot\Gamma_0' \right),
\end{equation*}
%
\begin{equation*}
\frac{d}{d\varepsilon}|\Gamma_\varepsilon'| |_{\varepsilon=0} =
\frac{ \left( (x' + \nabla u \cdot \Gamma_0')\nabla\phi\cdot\Gamma_0' +
(y' + \nabla v \cdot \Gamma_0')\nabla\psi\cdot\Gamma_0' \right) }
{\left( |x' + \nabla u \cdot \Gamma_0'|^2 + |y' + \nabla v \cdot \Gamma_0'|^2  \right)^\frac{1}{2}},
\end{equation*}
%
whence we obtain the first variation of $E(\Gamma_u)$
%
\begin{equation}\label{E:displaced_curve_1st_variation}
\delta E(\Gamma_u) = \int_{s_1}^{s_2}
\frac{ \left( (x' + \nabla u \cdot \Gamma_0')\nabla\phi\cdot\Gamma_0' +
(y' + \nabla v \cdot \Gamma_0')\nabla\psi\cdot\Gamma_0' \right) }
{\left( |x' + \nabla u \cdot \Gamma_0'|^2 + |y' + \nabla v \cdot \Gamma_0'|^2  \right)^\frac{1}{2}} ds.
\end{equation}
%
Note $(\phi)' = (\phi(x(s),y(s)))'=\nabla\phi\cdot\Gamma_0'$, $(\psi)'=\nabla\psi\cdot\Gamma_0'$. 

We integrate $(\phi)',(\psi)'$ by parts as we did for $E(\Gamma_0)$ and obtain
%
\begin{equation*}
\delta E(\Gamma_u) = \int \kappa_u(s) \vec{\varphi}(x,y) \cdot \hat{n}_u |\Gamma_u'(s)| ds,
\end{equation*}
%
where $\vec{\varphi}(x,y)=(\phi(x,y),\psi(x,y))$ and
\[
\hat{n}_u(s)=\frac{1}{|\Gamma_u'(s)|} (-y_u'(s),x_u'(s)), \qquad
\kappa_u(s) = \frac{x_u' y_u'' - x_u'' y_u'}{\left( |x_u'|^2 + |y_u'|^2 \right)^{\frac{1}{2}}}. 
\]

Let us rearrange the terms in \eqref{E:displaced_curve_1st_variation}.
First recall that the tangent vector $\hat{\tau}$ is given by
\[
\hat{tau} = \frac{\Gamma_0'}{|\Gamma_0'|} = \frac{1}{|\Gamma_0'|} (x',y'), \qquad 
\hat{tau} = (\tau_x,\tau_y).
\]
Rewrite $\delta E(\Gamma_u)$  in \eqref{E:displaced_curve_1st_variation}:
%
\begin{align*}
\delta E(\Gamma_u) &= \int
\frac{|\Gamma_0'|^2}{|\Gamma_0'|}
\frac{ \frac{x'}{|\Gamma_0'|} \nabla\phi \cdot \frac{\Gamma_0' }{|\Gamma_0'|}
+ \frac{y'}{|\Gamma_0'|} \nabla\psi \cdot \frac{\Gamma_0' }{|\Gamma_0'|}
+ \nabla u \cdot \frac{\Gamma_0'}{|\Gamma_0'|} \nabla\phi \cdot \frac{\Gamma_0'}{|\Gamma_0'|} 
+ \nabla v \cdot \frac{\Gamma_0'}{|\Gamma_0'|} \nabla\psi \cdot \frac{\Gamma_0'}{|\Gamma_0'|} }
{\left[ \left( \frac{x'}{|\Gamma_0'|} + \nabla u \cdot \frac{\Gamma_0'}{|\Gamma_0'|} \right)^2 +          
\left( \frac{y'}{|\Gamma_0'|} + \nabla v \cdot \frac{\Gamma_0'}{|\Gamma_0'|} \right)^2  \right]^\frac{1}{2}} ds \\
\delta E(\Gamma_u) &= \int
\frac{\tau_x \nabla\phi \cdot \hat{\tau} + \tau_y \nabla\psi \cdot \hat{\tau}
+ \nabla u \cdot \hat{\tau} \nabla\phi \cdot \hat{\tau}
+ \nabla v \cdot \hat{\tau} \nabla\psi \cdot \hat{\tau}}
{\left[ (\tau_x + \nabla u \cdot \hat{\tau})^2 + (\tau_y + \nabla v \cdot \hat{\tau})^2 \right]} ds.
\end{align*}
%
This can be written compactly as
\[
\delta E(\Gamma_u) = \int
\frac{\hat{\tau}^T (I + D\vec{u})^T D\vec{\varphi} \hat{\tau}}{|(I+D\vec{u})\hat{\tau}|} ds.
\]

\end{document}
